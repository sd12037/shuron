\section{\mc まとめ}
第1章でBCIに期待される社会的応用と、
運動想起型BCIに対する研究背景を述べた。
第2章と第3章では、現状多くの特徴量抽出手法や機械学習手法が試されている中で、
決定的なBCIのアーキテクチャが生まれていないことについて触れた。
第4章ではEEGの計測を行い、計測データからERDを検知する手法について提案した。
またERDに基づいた特徴量から足動作を検知する分類器について比較と検討を行った。
しかし、脳波解析に基づいてBCIを設計する場合にはユーザーの増大には対応が難しいと考えた。
第5章以降は、今後のBCIの社会応用に向けたEnd to End学習について提案と実験を行った。
BCIのアーキテクチャとしてニューラルネットワークを用いることとし、
特に以下の2つの項目を強調した。
\begin{itemize}
    % \item タスク毎のEEGの解析の必要性を排除
    \item 個人差におけるEEGの解析の必要性を排除
    \item 人類に共通した一般的なBCIの基本モデル構築
    % \item 類似タスクに関する学習の効率化
\end{itemize}
まず第一の結論として、個々人のEEGのみを収集し
ユーザーに対してオーダーメイドでBCIを構築するようなケースでは、
性能面とモデルの冗長性による学習困難を考慮し、End to End学習
によるメリットは現状ではほとんど存在しないと言える。
このような社会応用の場としては医療用途に特化しEEGの専門家が患者のEEGを解析できるような
場面が想定でき、本論文の第4章で示したような手法が適していると考える。

一方で社会応用を考える上では設計済のBCIを再利用することは必須であると思われる。
ユーザーが増加した場合に一人ひとりのEEGを解析しBCIを構築することは困難であるためである。
本研究の第二の結論として多数の被験者のEEGで学習を行うことにより、
新規の被験者に対しても従来手法並の性能を発揮できることが確認された。
このことは、個人事のEEGの解析や設計の必要性を排除することに直接的に貢献できる。

次に、多数の被験者のEEGにより学習を行ったBCIに対して、
個人向けに特化させることを想定し転移学習による実験を試みた。
結果としては、転移学習によって性能が向上する者と悪化する者が同等数見受けられた。
転移学習では学習済モデルが扱うタスクに近いタスクでなければ学習が上手く行かないが、
今回の場合はタスクは同じで被験者が全くの新規であるため、学習に参加した被験者の中に
EEGが似た人が含まれていることが焦点となりうる。


また、同様にして``人類に共通した一般的なBCIの基本モデル構築''の可能性を示唆した。

\section{\mc 今後の展望}
本研究では他分野のニューラルネットワーク応用に比べ、比較的簡易な
ニューラルネットワークの構造を採用した。
理由としてはEnd to Endの学習が可能であるかを検討し、
EEGへの応用の裾野を広げたいと考えていたためである。
従って、性能を追求した複雑かつ巨大なニューラルネットワークについては実験を行っていない。
ある特定の人物と特定の問題に特化させる場合には更なる工夫が必要であると考えられる。
しかし、本研究が示唆する成果によれば、
多様性のあるデータで学習した後に特定の問題に調整し直す転移学習が
BCIでも可能であると推察される。
従って特化型のBCIを考える場合においても、転移学習の可能性を模索していくことが重要だと思われる。
