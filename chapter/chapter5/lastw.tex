\section{\mc まとめ}
第1章でBCIに期待される社会的応用と、
運動想起型BCIに対する研究背景を述べた。
現状多くの特徴量抽出手法や機械学習手法が試されている中で、
FBCSPを除く決定的なBCIのアーキテクチャが生まれていないことについて触れた。
本研究ではBCIのアーキテクチャとしてニューラルネットワークを用いることを提案し、
特に以下の2つの項目を強調した。
\begin{itemize}
    % \item タスク毎のEEGの解析の必要性を排除
    \item 個人差におけるEEGの解析の必要性を排除
    \item 人類に共通した一般的なBCIの基本モデル構築
    % \item 類似タスクに関する学習の効率化
\end{itemize}

実験結果としては、ニューラルネットワークの構造を変更しなくとも
複数の問題について学習によって高い性能を獲得可能かを調べたが、
結果として満足のいくものではなかった。
しかし、この結果はニューラルネットワークの構造上の問題ではなく
データ量が少ないことによる未学習であると考え、複数の被験者のデータを統合することを試みた。
その結果、左手と右手の分類問題に関して学習が可能であることを示した。
同時に、新規被験者のデータに対しても新たな学習なしに一定レベルの正解率を得られることから
``個人差におけるEEGの解析の必要性を排除''に関して
ニューラルネットワークが大いに期待できることを示した。
また、同様にして``人類に共通した一般的なBCIの基本モデル構築''の可能性を示唆した。

\section{\mc 今後の展望}
本研究では他分野のニューラルネットワーク応用に比べ、比較的簡易な
ニューラルネットワークの構造を採用した。
理由としてはEnd to Endの学習が可能であるかを検討し、
EEGへの応用の裾野を広げたいと考えていたためである。
従って、性能を追求した複雑かつ巨大なニューラルネットワークについては実験を行っていない。
ある特定の人物と特定の問題に特化させる場合には更なる工夫が必要であると考えられる。
しかし、本研究が示唆する成果によれば、
多様性のあるデータで学習した後に特定の問題に調整し直す転移学習が
BCIでも可能であると推察される。
従って特化型のBCIを考える場合においても、転移学習の可能性を模索していくことが重要だと思われる。
