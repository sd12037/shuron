\section{\mc 提案手法の評価}
\subsection{\rm EEG \mc データ}
実験データとしてBCI2000システム\cite{BCI2000}を用いて記録された
PhysioNet\cite{PhysioNet}が提供する運動想起データセットを用いた。
電極の配置は標準的な10-10システム(図\ref{fig:10system})であり、サンプリング周波数は160Hzである。
109人の被験者が定められたタイムスケジュールに従い、
左手、右手、両手および両足の運動想起を行っている。
運動想起のタイムスケジュールを表す図を図\ref{fig:motorimage}に示す。
運動想起の開始時には被験者の前面に配置されたディスプレイから
右手あるいは左手の運動想起の指示が出される。
被験者は4秒間指示された手の
指を開いたり閉じたりする運動想起試行を続け、その後4.2秒間の休息時間が与えられる。
休息時間を終えると再びディスプレイから両手あるいは両足の運動想起の指示が出され、
被験者は4秒間の支持された部位の指を開いたり閉じたりする運動想起試行を行う。その後4.2秒の休息時間が再び与えられる。
計16.4秒を1サイクルとし、被験者1人につき45サイクルを繰り返し行う。
従って被験者は左手、右手、両手、両足で計90回の運動想起を行っている。

PhysioNetが提供する運動想起時のEEGデータベースは、
世界最大規模の被験者数を有している。
また、EEGの計測時に用いられたBCI2000システムは運動想起型BCIのみならず、
多種多様なBCIシステムを構築するために公開されたプログラム群であり、
多くの脳研究者に利用されている。
従って、手法の評価と検討を行うためのデータとして信頼性が高いと考えたため利用した。

\begin{figure}[t]
    \centering
    \includegraphics[width=12cm]{images/system1010.png}
    \caption{10-10システムの電極配置}
    \label{fig:10system}
\end{figure}
\begin{figure}[t]
    \centering
    \includegraphics[width=12cm]{images/motorimage.png}
    \caption{運動想起EEG測定のタイムスケジュール}
    \label{fig:motorimage}
\end{figure}


\subsection{\mc 評価方法}
本研究では各運動想起時間である4秒間のデータを取り出して、
2クラス分類の問題を\(_4C_2=6\)種類準備した。
ここで運動想起は一人あたり計90回行っており4部位あるため、
1つの部位につき22回あるいは23回の試行が行われている。
従って準備した2クラス問題におけるデータ数は44〜46となっている。
また、以降2クラス分類の問題は表\ref{table:pattern}に示す通り表記する。
\begin{table}[t]
    \centering
    \caption{分類問題の種類と論文内の表記}
    \begin{tabular}{|c|c|} \hline
        分類問題 & 論文内の表記 \\ \hline
        左手or右手 & LR \\ \hline
        両手or両足 & HF \\ \hline
        左手or両手 & LH \\ \hline
        左手or両足 & LF \\ \hline
        右手or両手 & RH \\ \hline
        右手or両足 & RF \\ \hline
    \end{tabular}
    \label{table:pattern}
\end{table}
問題LRを例にする。
ニューラルネットワークは4秒間のEEGデータを受け取り、
LかRかのいずれかを出力するように構成する。
EEGデータの与え方及びニューラルネットワークの構成を
被験者や問題ごとに変更を行わないことで、
単一のニューラルネットワークが被験者に関わらず、また
問題に関わらずBCIを構成可能かを評価する。

また評価指標としては正解率を用いる。
正解率はニューラルネットワークがEEGを元に出力する分類結果と、
EEG計測時に被験者が行った運動想起が一致している割合である。
また、評価に用いられるEEGは学習には用いていないデータである。
被験者は99人とし、それぞれ10-交差検証によって算出する。
各問題について99人の被験者の正解率の平均を取ることで、
単一のニューラルネットワークが各問題に適応可能かを評価する。





\subsection{\mc 結果}
各分類問題における正解率のヒストグラムを示す(図\ref{fig:LRhist}-\ref{fig:RFhist})。
横軸が正解率で、縦軸が被験者の数である。
どの分類問題においても60〜70\%付近の正解率となる被験者が多い。
一方で、80\%以上の正解率となる被験者から50\%の正解率となる被験者までいる。

\begin{figure}[t]
    \begin{minipage}{0.5\hsize}
     \begin{center}
      \includegraphics[width=65mm]{images/LR.png}
     \end{center}
     \caption{LR問題の正解率ヒストグラム}
     \label{fig:LRhist}
    \end{minipage}
    \begin{minipage}{0.5\hsize}
     \begin{center}
      \includegraphics[width=65mm]{images/HF.png}
     \end{center}
     \caption{LR問題の正解率ヒストグラム}
     \label{fig:HFhist}
    \end{minipage}
\end{figure}
\begin{figure}[t]
    \begin{minipage}{0.5\hsize}
     \begin{center}
      \includegraphics[width=65mm]{images/LH.png}
     \end{center}
     \caption{LH問題の正解率ヒストグラム}
     \label{fig:LHhist}
    \end{minipage}
    \begin{minipage}{0.5\hsize}
     \begin{center}
      \includegraphics[width=65mm]{images/RH.png}
     \end{center}
     \caption{RH問題の正解率ヒストグラム}
     \label{fig:RHhist}
    \end{minipage}
\end{figure}
\begin{figure}[t]
    \begin{minipage}{0.5\hsize}
     \begin{center}
      \includegraphics[width=65mm]{images/LF.png}
     \end{center}
     \caption{LF問題の正解率ヒストグラム}
     \label{fig:LFhist}
    \end{minipage}
    \begin{minipage}{0.5\hsize}
     \begin{center}
      \includegraphics[width=65mm]{images/RF.png}
     \end{center}
     \caption{RF問題の正解率ヒストグラム}
     \label{fig:RFhist}
    \end{minipage}
\end{figure}

各分類問題における全被験者の平均を表\ref{table:meanacc}に示す。
分類問題毎の平均の差異は小さく、タスクに依らず一定の性能を確保できることが分かる。
しかし、この結果はバタワースバンドパスフィルタと
CSPとLDAを組み合わせた手法\cite{benth}(表\ref{table:meanacc_csp})に対して
性能が高いとは言い難い。
\begin{table}[t]
    \begin{minipage}{0.5\hsize}
        \centering
        \caption{提案手法の問題毎の平均}
        \begin{tabular}{|c|c|} \hline
            分類問題 & 平均正解率 \\ \hline
            LR & 0.6712 \\ \hline
            HF & 0.6636 \\ \hline
            LH & 0.6322 \\ \hline
            LF & 0.6643 \\ \hline
            RH & 0.6517 \\ \hline
            RF & 0.6427 \\ \hline
        \end{tabular}
        \label{table:meanacc}
    \end{minipage}
    \begin{minipage}{0.5\hsize}
        \centering
        \caption{従来手法の問題毎の平均\cite{benth}}
        \begin{tabular}{|c|c|} \hline
            分類問題 & 平均正解率 \\ \hline
            LR & 0.6124 \\ \hline
            HF & 0.6701 \\ \hline
            LH & 0.7719 \\ \hline
            LF & 0.8093 \\ \hline
            RH & 0.7760 \\ \hline
            RF & 0.8196 \\ \hline
        \end{tabular}
        \label{table:meanacc_csp}
    \end{minipage}
\end{table}

\subsection{\mc 考察}
表\ref{table:meanacc} によると提案手法1のニューラルネットワークは
いずれの問題に対しても平均的には同等の対応力を有するように捉えられる。
しかし図\ref{fig:LRhist}-\ref{fig:RFhist}からは、
問題ごとに異なったヒストグラムが得られており任意のタスクに対して
同等に学習が可能であるとは言い難い。
また、問題LR以外は従来手法よりも性能が劣っている。

このような結果となった理由として、
実験に対して一回の学習に用いるデータ数が高々40個程であったことが
問題であると考える。
従来手法の最適化は第2章で述べた通り、解析的に実行が可能である。
また凸最適化問題であるため、BCIの構成要素1つ1つは
仮定した関数族の中で最適であることが保証されている。
従って、与えられた分類問題に有効な入出力関係を包含する関数族となっていれば、
極めて高い性能を保証することができる。
一方でニューラルネットワークの学習は一般的に非凸の最適化問題であり、
逐次的な最適化方法を用いなければならない。
学習データが少ない場合は逐次最適化で得られる勾配のバリエーションが乏しくなり、
探索範囲が極めて狭くなってしまうと推察できる。

一般にニューラルネットワークが表現できる関数の集合は非常に大きい。
すなわち提案した手法は分類問題に有効な入出力関係を包含する関数族となっているはずである。
しかし、学習データが少ない場合には大きな関数の集合の中を十分に探索することができない。

従って特定のタスクに対する個人のためのBCIを、ごく少数の学習データによって構築する場合には
ニューラルネットワークを用いる方法は不適だと考えられる。

\section{\mc 被験者のデータを混合したBCI構築}
\subsection{\mc データの混合}
個別のBCIに関しては学習データの収集自体が困難であるため、
パラメータを膨大に持つEnd to End学習は適していない。
一方で、社会応用を考える上ではユーザー毎に専門家がEEG解析を行うことは難しいため、
汎用的に利用できるBCIが求められる。

実験では従来手法として性能が良くなかったLR問題とHF問題を扱った。
まず、LR問題のデータを50人分集めすべてのデータを統合し、2200個へと増加させた。
すなわち左手の運動想起が1100回分と右手の運動想起が1100回分準備されたことになる。
実際には質の悪いEEGに関しては排除したため、
右手のデータを1042個、左手のデータを1050個用いることにし10-交差検証によって正解率を算出した。

\subsection{\mc 混合データに対する結果}
通常、ニューラルネットワークは交差検証は時間の都合上行わず
ホールドアウト法(ランダムに検証データを選んで、1度のみの検証を行う)を用いるが、
提案手法1をLR問題に用いた場合のみ交差検証を行った。
またニューラルネットワークの構造やハイパーパラメータの調整は行っておらず、
個人事に学習を行った際に使ったニューラルネットと全く同等のモデルを用いている。
結果として交差検証の平均正解率は0.8032となった。

表\ref{table:allvalidation}に10回の検証のそれぞれの正解率を示す。
各検証では50人の被験者のデータからランダムに220個のデータが検証データに用いられ、
残りの1980個のデータが学習データとして用いられている。
どのような組み合わせで学習が行われても75\%以上の高い性能を発揮していることが\ref{table:allvalidation}から分かる。
\begin{table}[t]
    \centering
    \caption{10交差検証の内訳}
    \scalebox{0.85}{
    \begin{tabular}{|c|c|c|c|c|c|c|c|c|c|} \hline
        検証1 & 検証2 & 検証3 & 検証4 & 検証5 &
        検証6 & 検証7 & 検証8 & 検証9 & 検証10 \\ \hline
        0.8318 & 0.8136 & 0.7909 & 0.7955 & 0.7909 &
        0.7500 & 0.8090 & 0.8318 & 0.8045 & 0.8136 \\ \hline
    \end{tabular}
    }
    \label{table:allvalidation}
\end{table}

学習時のAccuracyの変化を図\ref{fig:overfit1}に示す。
訓練データに対するAccuracyは学習epochが進む毎に増加傾向にあるにも関わらず、
検証データに対するAccuracyは学習epochが進むに連れて減少していく様子が分かる。
現状では過学習に陥ってしまっていることが確認される。
\begin{figure}[t]
    \centering
    \includegraphics[width=10cm]{images/linearof.png}
    \caption{過学習の様子:活性関数を用いない場合}
    \label{fig:overfit1}
\end{figure}


\subsection{\mc 活性化関数に \rm elu \mc を用いた場合の結果}
ここまでは従来手法の多くが線形変換の組み合わせによってBCIを構築してきたことを
意識し、提案手法にも非線形の活性化関数を導入していなかった。
ここで実験的に提案手法1のニューラルネットワークのすべての畳込み層の後にelu関数を追加した結果を示す。
画像認識などでは画像が本質的に非負値であることも相まって、
学習が比較的素早く行われるReLUを用いるが、EEGは時系列データであるため
ReLU関数を負の値まで拡張したelu関数を実験的に用いた。
その他の変更は行っていない。

学習は極めて素早く収束したことが同時に過学習も起こっており、
モデルの表現力としては既に十分であることが予測できる(図\ref{fig:overfit2})。

\begin{figure}[t]
    \centering
    \includegraphics[width=10cm]{images/overfitting.png}
    \caption{過学習の様子:elu活性化関数を用いた場合}
    \label{fig:overfit2}
\end{figure}

% \begin{figure}[t]
%     \begin{minipage}{0.5\hsize}
%         \centering
%         \includegraphics[width=6cm]{images/overfitting.png}
%         \caption{過学習の様子:elu活性化関数を用いた場合}
%         \label{fig:overfit}
%     \end{minipage}
%     \begin{minipage}{0.5\hsize}
%         \centering
%         \includegraphics[width=6.4cm]{images/linearof.png}
%         \caption{過学習の様子:活性関数を用いない場合}
%         \label{fig:overfit}
%     \end{minipage}
% \end{figure}

% \subsection{\mc 考察}
% 結果は個々の人間に対して学習させた正解率の結果よりも向上している。
% すなわち、学習データが多い場合には、EEGに対してもニューラルネットワークでEnd to Endの学習が可能であると言える。

% 非線形活性化関数を用いない場合にはニューラルネットワークは線形変換に制限されるため、
% その結果として過学習も引き起こりにくくなるはずであるが、実際には過学習している。
% 過学習の問題はアーキテクチャが決まった段階において試行錯誤的に対応する必要があるため、
% 結果として過学習が起こりやすくなるが、今回の場合は過学習した結果検証データに対して
% 著しく性能が低くなることはなかった。
% 活性化関数を用いない方法と正解率は僅差であるので、
% 適切な正則化によって非線形の表現力を維持しながら過学習を抑えるという方向性が有効だと言える。

% 多数のデータを用いることでニューラルネットが学習可能になったが、
% 多くの被験者の学習データを用いて構築されたニューラルネットワークが
% 社会的に大きなメリットを有するためには、新規の被験者へ転用できることが望ましい。
% 従って以下の実験を実施し、新規の被験者への適用が可能であるかを検討した。

\section{\mc 新規被験者への適用}
% \subsection{\mc 実験内容と結果}
ニューラルネットワークの学習データに
用いた50人の被験者を、以後トレーナーと記す。
またトレーナー以外の被験者49人をテスターと記す。
ニューラルネットワークに対してテスターのEEGデータを入力し、その正解率を算出した。
こちらのデータに対しては、ニューラルネットワークは全く未知の人間のEEGを受け取ることとなる。

% テスターとトレーナーの正解率のヒストグラムを図\ref{fig:tesacc}-\ref{fig:tracc}に示す。
% 図の通りトレーナーとテスターではトレーナーの正解率のほうが比較的高い。
% これは学習に参加している被験者であるため当然の結果と言える。
% しかし、テスターのヒストグラム図\ref{fig:tesacc}を図\ref{fig:croshis}とを比較した場合に、
% 図\ref{fig:tesacc}の方が比較的高い正解率を有する被験者が多いことが分かる。
テスターのデータに対する正解率の平均値は79.21\%であり、
テスターに対して個別に学習を行った際の交差検証の正解率66.81\%よりも高い。
しかし、図\ref{fig:tesacc}の通り正解率が0.5付近の被験者が2名存在しており、
全く適用できない被験者も存在する。
\begin{figure}[p]
    \centering
    \includegraphics[width=10cm]{images/tester_acc.png}
    \caption{テスターの正解率ヒストグラム}
    \label{fig:tesacc}
\end{figure}
% \begin{figure}[p]
%     \centering
%     \includegraphics[width=10cm]{images/traineracc.png}
%     \caption{トレーナーの正解率ヒストグラム}
%     \label{fig:tracc}
% \end{figure}
% \begin{figure}[p]
%     \centering
%     \includegraphics[width=10cm]{images/hikaku.png}
%     \caption{テスター毎に学習した際の正解率ヒストグラム}
%     \label{fig:croshis}
% \end{figure}

同様にしてHF問題についてもテストを行い、提案手法2についてもAccuracyを算出した。
結果をまとめて表\ref{table:kekka}に示す。
\begin{table}[t]
    \centering
    \caption{テストAccuracyの比較}
        \begin{tabular}{|c|c|c|c|} \hline
            問題\手法 & 提案手法1 & 提案手法2 & CSP + LDA \\ \hline
            LR &  79.21\%  & 76.78\%  & 61.24\%(99人個別設計)\\ \hline
            HF & 64.25\%  & 65.46\%  & 67.01\% (99人個別設計)\\ \hline
        \end{tabular}
    \label{table:kekka}
\end{table}

更に、トレーナーによって学習されたBCIを個々のテスター毎に最適化する転移学習を行った。
転移学習では各提案したニューラルネットワークの最後の層のみパラメータを調整し直した。
提案手法1のLR問題における転移学習の結果を図\ref{fig:trans1}に示し、
提案手法2のLR問題における転移学習の結果を図\ref{fig:trans2}に示す。
図の緑の点がトレーナーで学習したモデルを使って分類した際の各テスターのAccuracyである。
青の点は各テスター毎に転移学習を行った後のテストAccuracyであり、
著しくAccuracyが上昇する被験者から、80\%の正解率から60\%のAccuracyまで落ちている被験者もいる。

図\ref{fig:trans1}、\ref{fig:trans2}の緑の点(転移学習前のAccuracy)に着目すると、
いずれの手法を用いたとしてもsubject number10番の被験者は転移学習以前から
高いAccuracyとなっている。19番、35番も同様である。

\begin{figure}
    \centering
    \includegraphics[width=13cm]{images/trans_vs_not.png}
    \caption{提案手法1における転移学習の結果(青:転移学習後)}
    \label{fig:trans1}
\end{figure}
\begin{figure}
    \centering
    \includegraphics[width=13cm]{images/trans_vs_not_convLSTM.png}
    \caption{提案手法2における転移学習の結果(青:転移学習後)}
    \label{fig:trans2}
\end{figure}



\subsection{\mc 考察}
提案手法のいずれでも転移学習以前から高いAccuracyを有するテスターがいることは
、その被験者のEEGがトレーナーの有するEEGと元々似ているということが考えられる。
トレーナーと学習以前からAccuracyの高いテスターの間にある似ている特徴量が
どのようなものであるかを補足することが可能であれば、
End to End学習からEEGの新たな特徴量を発見することも考えうる。

また、テスターのデータに対する正解率が
従来手法を用いた個々の学習時よりも高かったことから、
多数のデータを用いて学習したニューラルネットワークは
運動想起実行時の共通したEEGの特徴を取り出すことができていると考えられる。


