\section{本論文の構成}
第1章では研究の背景と目的について述べた。
以降の章では以下の項目について説明する。
\begin{itemize}
    \item 第2章:
    基本的な手法を組み合わせた従来のBCIの構成について述べる。
    特にEEGの基礎研究でよく知られている現象を検知するための
    時間周波数解析に基づく方法と、BCIの分野で登場した信号処理手法である
    Common Spartial Patternを用いた手法について詳細に説明する。
    \item 第3章:
    BCIに頻繁に用いられる基礎的な信号処理・機械学習手法を説明する。
    また、EEGを実際に計測し、基礎的な手法の組み合わせによって
    EEGがどのように処理されるのかを確認した予備実験について述べる。
    \item 第4章:
    従来手法の問題点について再考し、
    End to End学習を目指したニューラルネットワークの提案モデルについて説明する。
    また、提案モデルに用いられるニューラルネットワークの基本について記述する。
    \item 第5章:
    提案手法を評価する複数の実験
    (1.個人のEEGに対するEnd to End学習、2.複数人にEEGに対するEnd to End学習、
    3.新規の人間のEEGに対する性能評価、4.新規の人間のEEGに対する学習済モデルの転移学習)
    の結果と考察について順次述べる。
    \item 第6章:
    実験を通して得られた結果から本研究の結論を述べ、
    今後の研究の発展性について記述する。
\end{itemize}
最後に本研究の遂行と論文の執筆を支援してくださった方への謝辞を述べ、
本論文の終了とする。