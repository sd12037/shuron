\section{研究の目的}
特徴量抽出手法や分類器を拡張する方針とは異なり、
より綿密にEEGの解析を行うことで性能の向上を達成しようとする試み続いている\cite{脳波解析BCI,脳波分析BCI,ERSBCI}。
EEGではBCIに限らず、頭皮領域と周波数帯域に関しての研究が盛んに行われてきたため、
特徴量抽出を行う場合にも電極と周波数に着目する場合が多い。
特にERDは特定の周波数における電位の減少であるため、
スペクトル解析と時間周波数解析が有効利用でき、
現在も運動想起型BCIのための研究が行われている\cite{時間周波数解析の比較}。
研究の多くがBCIあるいはEEGの解析に主眼が置かれているケースが多いため、
比較的BCIとしての性能は高くないか、BCIとしての評価がされない場合もある。
ただし、笹山の博士論文\cite{京大ドクター}に関してはCSPを用いた手法よりもERDに基づいたBCIが高い性能を発揮している
(EEGのみではなくMRIを用いた脳機能の複合的な解析を行い特徴量を抽出しているためと考えられる)。

現在、運動想起型BCIで高い性能を誇る手法が統計的信号処理や機械学習手法に基づいていることは不可解なことではない。
綿密なEEGの解析を実施するか否かに関わらず、
左手を動作させる場合と右手を動作させる場合とでは脳の活動は異なっているためである。
EEG解析の大きなモチベーションは、脳機能自体を解明したい場合か、多変量データを可視化したい場合である。
いずれの場合においても人間がデータを解釈するという点に比重が置かれていると言える。
一方でBCIというシステムを考える上では、人間がEEGの差異を明確に把握できる必要性は薄いと考えられる。
BCIが脳活動に基づいて動作していることを保証する必要はあるが、
本論文では、BCIが動作するための特徴量抽出を人間が目視できる形で与える必要はないと主張する。

一方で運動想起型BCIを構築するための特徴量抽出手法、分類手法は数多く存在する。
代表的なものを以下に記す。この中の幾つかは\ref{chapter:BCIのための要素技術}にて解説する。
\begin{itemize}
    \item Laplacian Filter(LF)
    \item Principal Component Analysis(PCA)
    \item Independent Component Analysis(ICA)
    \item Canonical Correspondence Analysis(CCA)
    \item Common Spartial Pattern(CSP)
    \item バンドパスフィルタバンク
    \item フーリエ変換
    \item ウェーブレット変換
    \item 自己回帰モデル
    \item Emperical Mode Decomposition(EMD)
    \item Linear Discriminat Analysis(LDA)
    \item Support Vector Machine(SVM)
    \item Logistic Regression(LR)
\end{itemize}
これらの手法はBCIの構成要素として組み込まれるが
数多くある手法の中から個々人に応じて、
あるいはタスクに応じて適切に組み合わせるのは容易ではない。
その中、FBCSPはバンドパスフィルタバンクとCSP
及びLDAを組み合わせた数少ない成功例であり、
現在でもバンドパスフィルタとCSPを基本とした派生手法が提案され続けている\cite{sparsemethod,bootCSP}。
しかし、数多くの手法が提案されている中で、
FBCSP以降は運動想起型BCIとしての決定的なアーキテクチャは提案されていないと言える。

本研究では音声認識や画像認識の分野で高い性能を発揮している
ニューラルネットワークに着目する。
ニューラルネットワークは学習時の計算量が膨大であることから、
長い間研究が滞っていたが、近年の計算機の発展により応用研究に用いられるようになり、
その際立った性能の高さから注目が集まっている。

ニューラルネットワークの実体は巨大な合成関数であり、
事実上単なるパラメトリックな数理モデルである。
更に、モデルの構築も含め、ハイパーパラメータの存在によって
試行錯誤の必要性も非常に高い。
しかし、特徴量抽出手法や分類器自体を変更しながら様々な組み合わせを検討することに比べ、
ニューラルネットワークの調整は単純作業である。
また今後ハイパーパラメータの調整自体を自動化する、あるいは学習に組み込む方法も出現する可能性がある。
誤差逆伝播学習によって原理上極めて深い合成関数の形式であっても学習が可能であるため、
モデルの設計を適切に行うことで、特徴抽出と分類を同時に達成できる可能性がある。
更にその学習のアルゴリズム自体は極めて単純であるが効果的に働き\cite{CheapLearning}、
再学習可能であるため有用なニューラルネットワークの構造が発見された場合にその再利用性が高い。
加えて、ニューラルネットワークの深層構造によって高い性能が引き出されること\cite{DeepvsShallow}や、
モデルの複雑さと比較して最適化問題の目的関数は病理的ではない、
あるいは病理的な形状を回避できることが示唆されるようになった\cite{ディープローカルミニマム}。

従って、運動想起型BCIに適したニューラルネットワークの構造を検討し、
基礎的な構造を確立することは有益であり、本研究の目的とする。
