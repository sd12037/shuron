\section{研究背景}

Brain Computer Interface(BCI)とは、脳とコンピュータを何らかの方法で接続し脳活動とコンピュータの動作を関連付ける技術の総称である。
BCIは脳とコンピュータの接続方法に関して「侵襲式」と「非侵襲式」の2つに分類され、実応用を考える上では非常に重要な項目となる。
侵襲式と非侵襲式とは、BCIを構築する際に外科手術を要するか否かを指す言葉であり、非侵襲式では外科手術を伴わない接続方法を用いてBCIを構築する。

非侵襲式は侵襲式に比べ人間への負担が明らかに少なく、BCIの実用化において大きなアドバンテージを有する。一方で、脳活動の計測精度という観点では劣り、BCIの構築のためには多くの工学的問題を解決する必要がある。
具体的には、計測時に混入するノイズの除去の問題や、脳活動の信号源の推定の問題を解決しなければならないと考えられている。
本稿では特に脳活動を頭皮上の電位変化であるElectroencephalogram(EEG)として計測し、計測データに応じてコンピュータが特定の処理を行うタイプのBCIに着目する。


\section{BCIの概要}
\subsection{外部刺激型BCI}
\subsection{運動想起型BCI}
本研究で着目する運動想起型BCIとは、人間が肢体運動を能動的に想像したことを検知するBCIである。主な応用としては麻痺患者のリハビリテーションが想定されているが、
他にも肢体運動を外部機器の操作に対応するさせることで依り汎用的な応用が期待できる。

\section{研究目的}
運動想起型BCIは人間が自発的な活動を行った際に生じる脳波を検知する必要があるが、
外部刺激に対する脳の活動に比べて被験者による個人差が激しく、
検知アルゴリズムを人手で設計するには限界がある。
従って、各個人毎にBCIの設計をしなければならないケースにおいて
検知アルゴリズムを自動で獲得する手法を開発することは重要な課題である。
そこで本研究では、個々の脳波を事前に計測・収集しておくことで、
人手での難解な脳波解析を回避しつつ高い精度で運動想起を検知できる機会学習モデルの構築を行う。
特に近年、画像認識や音声認識の分野で高い性能を示している深層学習に着目し、
脳波からの運動意図推定に適したニューラルネットワーク開発を行う。