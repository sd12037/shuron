\section{研究背景}

約1世紀前、脳の電気活動の研究が最初に行われて以来、
臨床、診断、およびリハビリのための脳活動の解析と解読に大きな関心が寄せられている。
いくつかの研究では、脳から放出される電気信号の特性は、
各脳活動および個々人に特有であることが示されている。
その結果、脳信号は以下のような領域で利用されている。
\begin{itemize}
    \item 医療応用:脳信号は、認知症やてんかん発作のような様々な精神障害の診断、および治療のために広く活用されている。
    さらに、多くの精神障害の早期診断のために脳信号が使用できることが示されている。
    \item 生体認証:脳信号は偽造、盗聴が困難であることから、固体の識別のための普遍的な生体情報となりうる。
    商用応用には不適な可能性があるが、他の生体信号と組み合わせることで認証システムの信頼性を向上することができる。
    \item Brain Computer Interface(BCI):脳信号はコンピュータや機械などの外界と、筋肉の動作無しに相互作用することができる。
    BCIは``direct neural interface''や``brain machine interface''とも表現され、基本的に脳と外部世界との間のインターフェースの役割を担う。
    脳の電気的活動を外部装置への制御信号に変換することで動作する。初期のBCIは、麻痺患者や障害のある患者が、車いす、義手義足、音声合成装置などの
    生活補助装置を制御することに役立つように設計された。しかし、近年は高度な精神的タスクを実行する健常者の支援や
    仮想空間での入力装置としての商用BCIが登場するに至っている。
\end{itemize}
BCIにおいて脳信号を計測する方法は、大きく分けて以下の3つがある。
\begin{itemize}
    \item 侵襲式:脳の灰白質へセンサを直接埋め込む方式
    \item 部分侵襲式:頭蓋骨の内部、脳の表面へセンサを埋め込む方式
    \item 非侵襲式:センサを頭蓋骨外部へ配置し、外科手術を必要としない方式
\end{itemize}
侵襲式と部分侵襲式ではセンサーを埋め込む外科手術を必要とし、
センサの寿命は数年しか続かないため、約2年毎に取り換えの手術を必要とする。
結果として侵襲式BCIの利用範囲は臨床試験に限られている。
対照的に、非侵襲式は外科手術の必要性もなく、商業及び医療的な用途の両方に用いられ発展する可能性がある。
非侵襲式は侵襲式に比べ人間への負担が明らかに少なく、BCIの実用化において大きなアドバンテージを有する。一方で、脳活動の計測精度という観点では劣り、BCIの構築のためには多くの工学的問題を解決する必要がある。
具体的には、計測時に混入するノイズの除去の問題や、脳活動の信号源の推定の問題を解決しなければならないと考えられている。
本稿では特に脳活動を頭皮上の電位変化であるElectroencephalogram(EEG)として計測し、計測データに応じてコンピュータが特定の処理を行うタイプのBCIに着目する。


\section{BCIの概要}
\subsection{外部刺激型BCI}
\subsection{運動想起型BCI}
本研究で着目する運動想起型BCIとは、人間が肢体運動を能動的に想像したことを検知するBCIである。主な応用としては麻痺患者のリハビリテーションが想定されているが、
他にも肢体運動を外部機器の操作に対応するさせることで依り汎用的な応用が期待できる。

\section{研究目的}
運動想起型BCIは人間が自発的な活動を行った際に生じる脳波を検知する必要があるが、
外部刺激に対する脳の活動に比べて被験者による個人差が激しく、
検知アルゴリズムを人手で設計するには限界がある。
従って、各個人毎にBCIの設計をしなければならないケースにおいて
検知アルゴリズムを自動で獲得する手法を開発することは重要な課題である。
そこで本研究では、個々の脳波を事前に計測・収集しておくことで、
人手での難解な脳波解析を回避しつつ高い精度で運動想起を検知できる機会学習モデルの構築を行う。
特に近年、画像認識や音声認識の分野で高い性能を示している深層学習に着目し、
脳波からの運動意図推定に適したニューラルネットワーク開発を行う。