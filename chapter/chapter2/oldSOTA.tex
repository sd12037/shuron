\section{従来手法のまとめ}
ここまで運動想起BCIを構築するための特徴量抽出や分類器について述べてきた。
具体的に運動想起BCIを構築する際には、様々な手法を適宜組み合わせて、
目的の性能を向上するために試行錯誤を重ねなければならない。
その際に考慮しなければならない脳波に関する知見を以下に記す。
\begin{itemize}
    \item 運動想起する身体部位に応じて、強く反応する頭皮領域は異なる。
    \item 事象関連脱同期は、特定の周波数領域に生ずる。
    \item 個人差や計測環境の影響を受けやすいため、ロバストな設計が必要となる。
\end{itemize}
上記の項目を考慮した上で運動想起BCIを構築するために以下の施策を取る。
\begin{itemize}
    \item 利用する電極の選定、あるいは空間フィルタなどによる電極の重み付けを行う。
    \item バンドパスフィルタによって特定の周波数波形のみを取り出す。
    \item 汎化性能の高い分類手法を用いる。
\end{itemize}
この章では、これらの施策を具体的に実行したBCI構築の手順例について紹介・検討し、
そこに生ずると考えられる問題点を把握し、研究のモチベーションを明らかにする。

\subsection{従来の運動想起BCIの構成と問題点}
まず典型的な運動想起BCIの構成について述べる。
運動想起を1回行った際に計測された脳波を以下で表記する。
\begin{equation}
    X =(x_1, \cdots, x_N)\in \mathbb{R}^{M \times N}\    
\end{equation}
ここに\(M\)は電極の個数、\(N\)は計測時間点数である。
典型的な運動想起BCIは、運動想起に関連している周波数帯域を取り出すためにバンドパスフィルタ\(\cal H\)
によって脳波の生データから\(\hat X(\cdot)\)を獲得することが一般的である。
\begin{equation}
    \hat X = {\cal H}(X)
    \label{eq:bandpass}
\end{equation}
次に、運動想起に関連している電極を選定するための空間フィルタ\(f(\cdot)\)を適用する。
\begin{equation}
    Z = f(\hat X)
    \label{eq:spatfilter}
\end{equation}
続いて重要な周波数帯域と頭皮領域について考慮がなされている\(Z\)に対して、
運動想起部位\(Y\)を出力する分類器\(g(\cdot)\)を準備することで、運動想起BCIが構成されている。
\begin{equation}
    Y = g(Z)
    \label{eq:classifier}
\end{equation}
このようにして構築されたBCIはバンドパスフィルタ\(\cal H\)、空間フィルタ\(f\)、分類器\(g\)
を要素とした集合として記述することができる。
\begin{equation}
    BCI=\{{\cal H},f,g\}
    \label{eq:BCI}
\end{equation}
また、時間周波数解析や行列分解・テンソル分解によって特徴量抽出を行う手法も提案されている。
しかし、BCIを構築する上では(あるいは脳波解析をする上では)周波数帯域と頭皮領域の考慮は欠かせないことであり、
バンドパスフィルタや空間フィルタは一般的に用いられる。
従って独自の特徴量抽出手法を\(h(\cdot)\)とすれば、
以下のように典型的なBCIに要素を追加した形で表記することができる。
\begin{equation}
    BCI=\{{\cal H},f,g,h\}
    \label{eq:generalBCI}
\end{equation}

以上の観点から、BCIを設計することは集合(\ref{eq:generalBCI})の要素をどのように決定するかを考えるということに相当する。
最も神経科学的知見に忠実な運動想起BCIは、
バンドパスフィルタ\(\cal H\)を事象関連脱同期の周波数帯域によって決定し、
空間フィルタ\(f\)を身体部位と対応する脳領域によって構成し、
処理された信号に対して閾値を設けることで分類器\(g\)を構築する。
機械学習の手法を取り入れた代表的なBCIは
バンドパスフィルタ\(\cal H\)を試行錯誤、あるいは何らかの評価指標に基づいて決定し、
空間フィルタ\(f\)をCSPによって構成し、
処理された信号に対してLDAなどを用いて分類器\(g\)を構築している。
更にCSSSPはバンドパスフィルタ\(\cal H\)と空間フィルタ\(f\)を同時に考慮した同時最適化手法、
FBCSPはバンドパスフィルタ\(\cal H\)を複数用意し、複数の周波数帯域から特徴量抽出を行う手法である。

従来手法に則ってBCIを設計することの問題点を検討する。
そのために典型的なワークフローの概略を示す。
まず、予め(\ref{eq:generalBCI})の要素を定める。
具体的にはバンドパスフィルタ\(\cal H\)をバタワースフィルタ、
空間フィルタ\(f\)をCSP、分類器\(g\)をLDAなどと定めることである。
この時に検討されるBCIを以下で表記する。
\begin{equation}
    BCI=\{{\cal H}_{buttord},f_{csp},g_{lda}\}
    \label{eq:a BCI}
\end{equation}
ここで(\ref{eq:a BCI})の各要素も実数スカラー、実数ベクトル、実数行列などを要素に持つ集合であり、
以下のように表される。
\begin{eqnarray}
    {\cal H}_{buttord} &=& \{w_p,w_s,r_p,r_s \mid w_p,w_s \in \mathbb R^2, r_p,r_s \in \mathbb R\} \\
    \label{eq:H_buttord}
    f_{csp} &=& \{W_{csp}\ \mid W_{csp} \in \mathbb R^{m\times M} \}\\
    \label{eq:f_csp}
    g_{lda} &=& \{W_{lda}\ \mid W_{lda} \in \mathbb R^{d \times m}\}    
    \label{eq:g_lda}
\end{eqnarray}
\(w_p,w_s\)はそれぞれ通過帯域コーナー周波数、阻止帯域コーナー周波数である。
\(r_p,r_s\)はそれぞれ通過帯域リップル、阻止帯域の減衰量である。
\(W_{csp}\)はCSPにおける線形変換の表現行列であり、
電極数\(M\)の次元を持つ脳波を\(m (< M)\)の任意の次元の電極空間に圧縮する。
\(W_{lda}\)はLDAにおける線形変換の表現行列であり、
\(m\)次元の電極空間から分類クラスの数より小さな任意の\(d\)次元への射影を担う。

改めて(\ref{eq:a BCI})についてその要素を書き下すと以下のように記述できる。
\begin{equation}
    BCI=\{\{w_p,w_s,r_p,r_s\},
    \{W_{csp}\},
    \{W_{lda}\}\}
    \label{eq:a BCI all}
\end{equation}
すなわちここで検討しているBCIは
(\ref{eq:a BCI all})の要素を決定することによって構築され、
第1の要素に含まれるパラメータがバンドパスフィルタを決定し、
第2の要素に含まれるパラメータが空間フィルタを決定し、
第3の要素に含まれるパラメータが分類器を決定する。
脳波処理を適切に行うための手法を柔軟に検討できるかのように見えるが、
ここで考慮したBCIには実際には明らかな制限が存在する。
CSPのパラメータの決定方法については\ref{subsec:CSP}で述べ、
同様にLDAのパラメータの決定方法については\ref{subsec:LDA}で述べた。
いずれも解析的に決定されるパラメータであることを考えると、
事実上、(\ref{eq:a BCI all})において自由に決定できるパラメータは
第1の要素に含まれているものだけであり、第2の要素と第3の要素は第1の要素を決めた時点で
(計算の必要性は生じるが)決定されている。
唯一、CSPとLDAにおける写像の次元についてだけは検討できるが、
これらは最適化問題を解いた後、いくつかの次元を無視することで達成されるのであって、
本質的に最適化問題の振る舞いが変わることはない。従って、ここまで検討してきたBCIに関しては
バンドパスフィルタの設計に極めて敏感に性能が左右されると想定される。
他方で、考慮しなければならないパラメータが極めて少ないために
設計が容易であると捉えることも可能であるが、
CSSSPやFBCSPなどがここで検討したBCIよりも高い性能を示していることから、
より大きな設計自由度を与える必要があると言える。