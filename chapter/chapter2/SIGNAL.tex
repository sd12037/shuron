
\section{\mc 周波数解析}
スカラーの時間波形\(x(t)\)に対して、周波数スペクトルを算出する際に用いられるフーリエ変換について述べる。
\(t\)が連続量である時\(x(t)\)をアナログ信号と呼び、離散的であればディジタル信号と呼ぶ。
このセクションではアナログ信号とディジタル信号が混在するため、
アナログ信号に関しては\(x(t)\)と表記し、ディジタル信号を\(x_n\)と表記することで明確に区別する。
\subsection{4\mc 種類のフーリエ変換}
時間を連続的に扱うか離散的に扱うかの違いだけでなく、
周波数を連続的に扱うか否かによってもフーリエ変換の式は異なっており、\(2\times 2 = 4\)つのフーリエ変換がある。
通常はフーリエ変換について議論する場合は時間も周波数も連続であるとみなした式を使う場合が多いが、
コンピュータ上で数値の処理を行う場合は原則離散的であるため、4つのフーリエ変換について全て簡単に説明する。

\subsubsection{\mc フーリエ変換}
スカラーの時間波形\(x(t)\)に対して、フーリエ変換は以下の(\ref{eq:FT})で表され、
逆フーリエ変換の式は(\ref{eq:iFT})で表される。
\begin{equation}
    X(\Omega)=\int_{-\infty}^{\infty} x(t)\exp(-i\Omega t)dt
    \label{eq:FT}
\end{equation}
\begin{equation}
    x(t)=\frac{1}{2\pi} \int_{-\infty}^{\infty} X(\Omega)\exp(i\Omega t)d\Omega
    \label{eq:iFT}
\end{equation}
ここで、\(i\)は虚数単位であり\(X(\Omega\))は\(\Omega\)を変数とするスカラー関数である。
逆フーリエ変換はスカラーの時間波形\(x(t)\)をある周波数\(\Omega\)の波形\(\exp(i\Omega t)\)の
線型結合によって表現した式である。このときの結合係数が\(X(\Omega)\)である。
工学的な立場では時間波形\(x(t)\)が与えられた時に、波形\(\exp(i\Omega t)\)による線型結合をすることで、
波形にどのような周波数の波形がどれくらいの割合で含まれているかを知りたいケースがある。
それを可能にするのがフーリエ変換であり、\(x(t)\)から結合係数\(X(\Omega)\)を算出することが可能である。
通常はこの時の結合係数\(X(\Omega)\)を周波数スペクトルと呼ぶ。
逆フーリエ変換(\ref{eq:iFT})に登場する定数倍\(1/2\pi\)は、フーリエ変換\(\cal F(\cdot)\)と逆フーリエ変換\({\cal F}^{-1}(\cdot)\)が、
フーリエ変換可能な時間波形\(x(t)\)に対して
\({\cal F}^{-1}({\cal F}(x(t))) = x(t)\)となるように調整するための係数である。
この係数は逆フーリエ変換の式ではなくフーリエ変換の式に付いていてもよく、あるいは両方の式に平方根の形で分配されていても構わない。
またフーリエ変換可能な時間波形とは以下を満たす\(x(t)\)である
\begin{equation}
    \int_{-\infty}^{\infty}|x(t)|dt < \infty
\end{equation}

\subsubsection{\mc 離散周波数フーリエ変換}
取りうる周波数を離散的にした場合のフーリエ変換について説明する。
離散周波数フーリエ変換という命名は便宜的にこの論文内で行っているものであり一般的ではない。
通常、ここで紹介するフーリエ変換は「フーリエ級数展開」として知られている。
歴史的には微分方程式を解くために開発され、フーリエ変換よりも先に発見されている。
離散周波数フーリエ変換と離散周波数逆フーリエ変換はそれぞれ(\ref{eq:sFT})と(\ref{eq:siFT})で表される。
\begin{equation}
    X_k=\frac{1}{T_0}\int_{-\frac{T_0}{2}}^{\frac{T_0}{2}} x(t)\exp(-i\Omega_0 kt)dt
    \label{eq:sFT}
\end{equation}
\begin{equation}
    x(t)=\sum_{k=-\infty}^{\infty} X_k \exp(i\Omega_0 k t)
    \label{eq:siFT}
\end{equation}
離散周波数逆フーリエ変換は、スカラーの時間波形\(x(t)\)を離散的な周波数\(\Omega_0 k\)の波形\(\exp(i\Omega_0 k t)\)の
線型結合によって表現した式である。このときの結合係数が\(X_k\)である。
ここに\(T_0\)は時間波形の周期であり、\(\Omega_0 = 2\pi/T_0\)を基本周波数と呼ぶ。
離散周波数フーリエ変換が、時間波形が与えられた時の周波数スペクトルを算出する役割を担うことは、フーリエ変換と同様である。
フーリエ変換との最たる違いは、時間波形に対して周期性を仮定している点であり、その周期は既知でなければならない。
時間波形が周期性を持つ場合には周波数は離散的な値を取る。

\subsubsection{\mc 離散時間フーリエ変換}
取りうる時間を離散的にした場合のフーリエ変換について説明する。
離散時間フーリエ変換と離散時間逆フーリエ変換はそれぞれ(\ref{eq:tFT})と(\ref{eq:tiFT})で表される。
\begin{equation}
    X(\omega)=\sum_{n = -\infty}^{\infty} x_n \exp(-i\omega n)
    \label{eq:tFT}
\end{equation}
\begin{equation}
    x_n=\frac{1}{2\pi} \int_{-\pi}^{\pi} X(\omega) \exp(i\omega n)d\omega
    \label{eq:tiFT}
\end{equation}
離散時間逆フーリエ変換が、時間波形\(x_n\)を周波数スペクトル\(X(\omega)\)を結合係数とした
\(\exp(i\omega n)\)の線型結合を表しているのはこれまでと同様である。
しかし、離散時間フーリエ変換では周波数スペクトル\(X(\omega)\)が周期的な関数となることは
強調しておかねばならない。

\subsubsection{\mc 離散フーリエ変換}
離散フーリエ変換は、取りうる時間も周波数も離散的であるとした場合のフーリエ変換である。
離散フーリエ変換と離散逆フーリエ変換はそれぞれ(\ref{eq:tsFT})と(\ref{eq:tsiFT})で表される。
\begin{equation}
    X_k=\sum_{n = 0}^{N-1} x_n \exp \left(-i\frac{2\pi}{N} k n \right)
    \label{eq:tsFT}
\end{equation}
\begin{equation}
    x_n=\frac{1}{N}\sum_{k=0}^{N-1} X_k \exp \left(i \frac{2\pi}{N} k n \right)
    \label{eq:tsiFT}
\end{equation}
ここで\(N\)はサンプル時間点数の意味で周期である。
時間波形に対する周期は\(N\)であるが、周波数スペクトルに対する周期も\(N\)となる。
仮に1周期が1000点のサンプル点によって構成される時間波形に対し離散フーリエ変換を用いた場合は、
得られる周波数スペクトルは1000点で周期を有する形式となる。
離散フーリエ変換は有限の数列から有限の数列への変換であり、他のフーリエ変換と異なり無限大を扱う必要はないため
コンピュータ上で計算を実行することが可能である。
高速フーリエ変換と呼ばれる実応用で頻繁に用いられるアルゴリズムは、
離散フーリエ変換(\ref{eq:tsFT})を高速に実行する手続きのことである。

\subsection{\mc パワースペクトル密度}
パワースペクトル密度\(PSD(\Omega)\)とは時間信号\(x(t)\)の周波数スペクトルを\(X(\Omega)\)とした時、
\begin{equation}
    PSD(\Omega)=|X(\Omega)|^2
    \label{period}
\end{equation}
に相当する関数である。
信号\(x(t)\)のエネルギーが周波数\(\Omega\)に関してどのように分布するかを示している。
コンピュータでパワースペクトル密度を計算する場合は(\ref{eq:tsFT})で計算される\(X_k\)
の二乗\(|X_k|^2\)を算出する。
しかし、数学的な定式化を行う上では以下の定義が用いられ、\(|X_k|^2\)はピリオドグラムと呼び区別する。
\begin{equation}
    PSD(\Omega) = \int_{-\infty}^{\infty} R(\tau)\exp(-i\Omega \tau)d\tau
    \label{eq:wh_theorem}
\end{equation}
ここに、\(R(\tau)\)は
\begin{equation}
    R(\tau)  =  \mathbb E[x(t)x(t+\tau)]    
\end{equation}
であり、信号\(x(t)\)の自己相関関数である。(\ref{eq:wh_theorem})は、ウィーナー・ヒンチンの定理としても知られており、
自己相関関数のフーリエ変換がパワースペクトル密度になることを示している。
この定義から明らかなようにパワースペクトル密度とは統計量であって、
解析的に算出されるのではなく、推定されるものである。

ピリオドグラムはパワースペクトル密度の推定を行う手段として用いられる。
一般的にピリオドグラムによる推定値は平均と標準偏差が同じ大きさを持ち、
時系列の長さを長くしても推定誤差は改善されない。
時系列を長くすることで周波数分解能を高くすることはできるが、
個々の周波数でのスペクトルの相対誤差は変化しない。
そこで、通常は何らかの平滑化を行って個々の周波数成分の推定値の誤差を減少させる方法が一般に採用される。
主な方法として、時間領域で波形を分割し、複数のピリオドグラムの平均を算出する方法がある。
時間領域で波形を分割する際には、波形が時間で互いに重なりを持つように分割する、Welchのオーバーラッピング法が用いられる。
また最大エントロピー法を用いてパワースペクトル密度を推定する方法もあり、
その場合には(\ref{eq:wh_theorem})を制約条件としたエントロピー最大化の変分問題を解くこととなる。
このパワースペクトル密度推定方法はバーグ法として知られている。
スペクトル解析に関する記述は書籍\cite{スペクトル解析}に詳しい。

脳波は、運動想起時に特定の周波数領域でエネルギーが減少する事象関連脱同期が生じるとされているため、
適切な電極選定を行い、
脳波のパワースペクトル密度\(PSD(\Omega)\)を推定することができれば、
とある\(\Omega\)で著しくパワーが減少する様子が確認できる。