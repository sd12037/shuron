
\section{フーリエ解析と時間周波数解析}

\subsection{周波数スペクトル}
周波数スペクトルは、脳波研究の知見を使った有効な特徴量である考えられる。
1章で述べたように、
運動想起時には想起部位に応じた頭皮上において、脳波の特定の周波数のパワーが減少する事象関連脱同期が生じることが知られている。
従って、脳波の周波数スペクトルは運動想起時と休息時で異なると期待され、良い特徴量になることが想定される。

この手法の主な流れを記す。まず脳波信号を\(x(t) \in \mathbb{R}^D\)と表記する。ここに、\(D\)は電極の個数である。
まず電極に対しての重み付けベクトル(空間フィルタ)である$w \in \mathbb{R}^D$を何らかの方法で決定することで、
\begin{equation}
    z(t) = w^T x(t) \in \mathbb{R}    
\end{equation}
とスカラー値の時間波形に変換する。空間フィルタとしては着目したい頭皮領域に対して``スモールラプラシアンフィルタ''のような
決定論的な手法を用いる場合もあれば、PCAやICA、また次節で述べるCommon Spatial Pattern(CSP)のような統計的な指標に基づいた手法を用いる場合もある。
次に$z(t)$に対してフーリエ変換${\cal F}(\cdot)$を行い、スペクトル
\begin{equation}
    A(f) = {\cal F}(z(t))
\end{equation}
を獲得する。運動想起が行われた場合には特定の周波数において、パワースペクトル\(A(f)^2\)の減少が生じると期待できる。

\subsection{スペクトログラム}
フーリエ変換の代わりに$z(t)$に対して短時間フーリエ変換${\cal STFT}(\cdot)$を行い、
\begin{equation}
    A(t,f) = {\cal STFT}(z(t))
\end{equation}
を獲得し、スペクトログラムを特徴量として用いることもできる。
運動想起開始から事象関連脱同期がどのタイミングでどのくらいの長さ継続するのかも反映することができる。
あるいはウェーブレット変換などの他の時間周波数解析を用いることも可能である。