%%%%%%%%%%%%%% 特徴量工学の説明 %%%%%%%%%%%%
\section{\mc 時間周波数解析の基づく手法}
運動想起型BCIではEEG信号から運動想起部位を識別することが目的となる。
既に運動想起時にはERDという現象が生ずることが知られているため\cite{ERDとERS}、
運動想起型BCIではERDを正確に検知することを主眼にした時間周波数解析を用いた研究が
多数存在することを\ref{MIBCIの周辺}で述べた。
以下、時間周波数解析に基づいた手法の基本形について述べる。

% 例として
% 図\ref{fig:EEGfootmove}と図\ref{fig:EEGhandmove}を添付する。
% それぞれ足の運動想起と手の運動想起を行った際のEEGである。
% いずれも0秒以前はディスプレイを注視した状態であり、0秒以降に運動想起を4秒間行っている。
% 電極の個数は64個用いられており、全ての電極の波形が表示されている。
% 足の運動想起が行われているのか手の運動想起が行われているのかを
% EEGの生データから識別するのは困難であることが分かる。
% \begin{figure}[t]
%     \centering
%     \includegraphics[width=9cm]{images/EEGfootmove.png}
%     \caption{0秒〜4秒間に足の運動想起を行った際のEEG}
%     \label{fig:EEGfootmove}
% \end{figure}
% \begin{figure}[t]
%     \centering
%     \includegraphics[width=9cm]{images/EEGhandmove.png}
%     \caption{0秒〜4秒間に手の運動想起を行った際のEEG}
%     \label{fig:EEGhandmove}
% \end{figure}
% そこで、何らかの変換\(f(\cdot)\)をEEG信号\(x(t)\)に対して施すことで、
% 特徴量\(z=f(x(t))\)の獲得を目指すのが特徴量工学である。
% この節ではまず運動想起に関するEEGの研究により得られた知見から特徴量を得る方法をいくつか紹介する。

\subsection{\mc 頭皮領域と空間フィルタ}
\subsubsection{\mc スモールラプラシアンとラージラプラシアン}
まずEEGを基にした運動想起型BCIではERDの検出に必要な電極の選定や
センサ領域への変換である空間フィルタに関して考慮するのが一般的であり、
代表的な空間フィルタとしてスモールラプラシアンフィルタとラージラプラシアンフィルタがある。
% 例えば足の動作に関する脳活動がEEGとして観測される場合、
% 頭皮上の全ての領域が対等に重要であるとは考えにくい。
% 足の動作に関する脳の領域は頭頂部に存在するため、EEGとして観測される場合、
% 頭頂部に位置するCz電極が最も関係していると考えられる。
% 図\ref{fig:Cz}は足の運動想起を行った際のCz電極によって記録されたEEGである。
% しかし波形から足の運動想起を行っていることを判別する決定的な差異を見出すことは困難である。
% \begin{figure}[t]
%     \centering
%     \includegraphics[width=9cm]{images/CzEEG.png}
%     \caption{0秒〜4秒間に足の運動想起を行った際のCz電極の波形}
%     \label{fig:Cz}
% \end{figure}
% 理由としては、脳の頭頂部での神経活動によって生じている電位が、
% 必ずしも頭皮上の頭頂部にのみ伝搬するとは限らないからである。
% 電位を発生させている神経活動と電位を計測している電極との間には、頭蓋骨と皮膚があり、
% 電位の伝搬がどのように行われているかを完全に把握することは難しい。
% ある程度空間的に広がりを持って電位が伝搬していると考えるのが自然である。
% そこで計測されたEEGに対して、
% スモールラプラシアンフィルタと呼ばれる空間フィルタが用いられることがある。
一般に2変数関数\(f(x,y)\)に関してラプラシアンは一般的に以下で表される。
\begin{eqnarray}
    \nabla^2 f(x,y) && =  \frac{\partial^2}{\partial x^2}f(x,y)+\frac{\partial^2}{\partial y^2}f(x,y) \nonumber \\
    && =  \frac{\partial}{\partial x}\left\{ \frac{f(x+dx, y) - f(x,y)}{dx} \right\}
     + \frac{\partial}{\partial y}\left\{ \frac{f(x, y+dy) - f(x,y)}{dy} \right\} \nonumber \\
    && =  \frac{ \{f(x+dx,y)-f(x,y)\} - \{f(x,y)-f(x-dx,y)\}}{dx^2} \nonumber \\
    && \quad \quad +  \frac{ \{f(x,y+dy)-f(x,y)\} - \{f(x,y)-f(x,y-dy)\}}{dy^2} \nonumber \\
    && =  \frac{1}{dr^2}\{f (x+dx,y)+f(x-dx,y) \nonumber \\
    && \quad \quad \quad \quad +  f(x,y+dy)+f(x,y-dy)-4f(x,y)\}
\end{eqnarray}
ここに\(dr^2=dx^2=dy^2\)である。この定義式について変数\(x,y\)を離散化すると以下の式となる。
\begin{equation}
    \nabla^2 f(x,y) = -4f(x,y) + f(x-1,y) + f(x+1,y) + f(x,y-1) + f(x,y+1)
\end{equation}
EEGに対するラプラシアンフィルタは上記の式の正負を反転し、
\(4\)で割った以下の式が用いられ、スモールラプラシアンと呼ばれている\cite{スモールラプラシアン}。
\begin{equation}
    {\rm SmallLap} f(x,y) = f(x,y) - \frac{1}{4} \{ f(x-1,y) + f(x+1,y) + f(x,y-1) + f(x,y+1) \}
\end{equation}
また、ラージラプラシアンと呼ばれる概念もあり\cite{スモールラプラシアン}、差分を取るインデックスを1つ飛ばした
以下の式が用いられる。
\begin{equation}
    {\rm LargeLap} f(x,y) = f(x,y) - \frac{1}{4} \{ f(x-1,y) + f(x+1,y) + f(x,y-1) + f(x,y+1) \}
\end{equation}
% 二次差分フィルタは画像処理では輪郭検出のために用いられる処理であり、
% 空間的に配置された値が際立って変化する位置で大きな値を持つ配列を返す。
例として、A電極によって計測されたEEGを\(x_A(t) \in \mathbb R\)と表記するとし、
Cz電極に対するスモールラプラシアンフィルタとラージラプラシアンフィルタ
が適用されたEEGをそれぞれ\(x_{smallCz}(t), x_{largeCz}(t)\)として、以下の式で表される(図\ref{fig:smalllap},\ref{fig:largelap})。
\begin{eqnarray}
    x_{smallCz}(t) & = & x_{Cz}(t) - \frac{1}{4}(x_{C1}(t) + x_{C2}(t) + x_{FCz}(t) + x_{CPz}(t)) \\
    x_{largeCz}(t) & = & x_{Cz}(t) - \frac{1}{4}(x_{C3}(t) + x_{C4}(t) + x_{Fz}(t) + x_{Pz}(t))     
\end{eqnarray}
\begin{figure}[p]
    \centering
    \includegraphics[width=9cm]{images/smalllap.png}
    \caption{Cz電極に対するスモールラプラシアンフィルタ}
    \label{fig:smalllap}
\end{figure}
\begin{figure}[p]
    \centering
    \includegraphics[width=9cm]{images/largelap.png}
    \caption{Cz電極に対するラージラプラシアンフィルタ}
    \label{fig:largelap}
\end{figure}
% 図\ref{fig:Cz}はCz電極に対してスモールラプラシアンフィルタを用いた際のEEGである。
% \begin{figure}[t]
%     \centering
%     \includegraphics[width=9cm]{images/rapCzEEG.png}
%     \caption{スモールラプラシアンフィルタを適用した際のCz電極の波形}
%     \label{fig:Cz}
% \end{figure}
\subsubsection{一般的な空間フィルタ}
スモールラプラシアンやラージラプラシアンの処理によって必ずしも有効な特徴量が得られるとは限らないため、
適切な空間フィルタを獲得するために統計的なフィルタ設計方法が用いられる場合もある。
% ここに画像処理で空間フィルタという表現を使う場合、2次元配列に関しての計算を考えるのが一般的であるが、
% EEGに対して空間フィルタという表現を用いる場合は、電極に対する重み付けを行う処理全般を指す。
電極をD個用いてEEG\(x(t) \in \mathbb R^D\)を計測した場合に、\(w \in \mathbb R^D\)によって
\begin{equation}
    z(t) = w^T x(t) \in \mathbb R
\end{equation}
を計算した場合は\(w\)を空間フィルタと呼ぶ。
より一般的には行列\(W = (w_1, w_2, \cdots, w_d) \in \mathbb R^{D \times d}\)によって、
\begin{equation}
    z(t) = W^T x(t) \in \mathbb R^d
\end{equation}
を計算する場合にも空間フィルタを用いたと言える。
この場合は、\(W\)の各列が空間フィルタ1つ1つに対応し、
\(d\)個の空間フィルタについて一挙に計算を行うことが可能である。
特にIndependet Componet Analysis(ICA)による空間フィルタの設計は
既にEEG解析で一定の地位を築いている。またEEG解析ソフトウェアの機能の1つとして実装されているケースが多く、
書籍\cite{脳波解析入門}では、1章を割いてICAの応用方法について解説がなされている。

より一般的に、多数のセンサ間(あるいは頭皮領域)にどのような関連があるかを明らかにするため、
東らの\cite{グラフフーリエ}では、グラフフーリエ領域と呼ばれる
信号構造で定義される領域へ信号を変換する方法を提案している。
この手法は、電極の配置における空間的距離と観測信号の類似性に基づいて、
電極間の繋がりの強さを示すグラフ構造を獲得する。
グラフフーリエ領域を用いて空間的な平滑化を行った後に、従来の信号源推定手法である
Principan Component Analysisなどを用いると、真の信号源に近い信号の抽出が可能であると主張されている。

\subsection{\mc 時間周波数解析}
% EEGにはδ波、θ波、α波、β波などが存在するが、これらは周波数帯域で分類がなされており
% (表\ref{table:eegtype})
% 、それぞれの波の位相や振幅が精神状態に応じて時間変化する。
% \begin{table}[tbp]
%     \centering
%     \caption{EEGの周波数よる分類}
%     \begin{tabular}{|c|c|c|} \hline
%         名称 & 読み & 周波数帯域 \\ 
%         \hline
%         δ波 & デルタ波 & 0.5〜4Hz \\ 
%         \hline
%         θ波 & シータ波 & 4〜8Hz \\ 
%         \hline
%         α波 & アルファ波 & 8〜13Hz \\ 
%         \hline
%         β波 & ベータ波 & 13〜Hz \\
%         \hline
%     \end{tabular}
%     \label{table:eegtype}
% \end{table}
% 特にヒトが手や指、足等の運動およびそのイメージを行う際に生ずる、
% α律動やβ律動でのERDおよびERSは特徴量になると考えられ、
% ERDは運動およびそのイメージ中に、ERSは運動およびそのイメージ後に発生する\cite{ERDとERS}。
空間フィルタによる処理を行ったEEGの
スペクトログラムを時間周波数解析によって獲得することで、
ERDを直接的に検知し、運動想起BCIを構築することが可能である\cite{Beta波によるBCI}。
ERDやERSを検知するための時間周波数解析には、最も単純なものとして短時間フーリエ変換を用いることができる。
しかし、フーリエ変換対に対する不確定性関係やスペクトル密度推定の精度の問題もあり
ウェーブレット変換を用いる方法やバーグ法に基づく方法も検討されている。
またERD検知に有効な時間周波数解析の比較についても研究報告がある\cite{時間周波数解析の比較}。
更に、非定常時系列に対する時間周波数解析として提案された
Empirical Mode Decomposition(EMD)を用いる方法もいくつか検討がなされている\cite{EMD,IMF}。

% いずれの時間周波数解析を用いるとしても、その変換\(TFA(\cdot)\)は
% 時間\(t\)の関数\(x(t)\)から、時間\(t\)と周波数\(f\)の関数\(X(t,f)\)への変換として表される。
% \begin{equation}
%     X(t,f) = TFA(x(t))
%     \label{eq:time-freq}
% \end{equation}
% この時、ERDが生じているのならば\(|X(t,f)|^2\)がある区間\([f_{low}, f_{high}],[t_{start}, t_{end}]\)で
% 著しく減少しているはずである。
% \(|X(t,f)|^2\)に顕著な変化を見出だせるような変換\(TFA\)を見つけ出すことが、
% 時間周波数領域における特徴量抽出に相当する。
% \section{その他の手法}
% \subsection{スペクトログラムを特徴量とする手法}
% 運動想起型BCIでは、上記までに述べてきたCSPによる手法が非常に活発に議論されている。
% しかし、CSPによる手法は計測の際に多数の電極を用いる必要がある。
% 一方で実用上は電極が少なければ少ない程計測における負担は少なくなるため、
% 少数の電極のみを用いてBCIを構築する方法も研究がされている。
% 例として運動時、あるいは運動想起時には特定の頭皮領域に配置された電極において、
% 特定の周波数帯域にパワーの減少が生じること(事象関連脱同期)が知られているため、
% この現象に狙いを定めてBCIを構築する研究も盛んである。

% この手の手法の主な流れを示す。まずEEG信号を\(X \in \mathbb{R}^{M \times N}\)と表記する。ここに、\(M\)は電極の個数、\(N\)は計測時間点数である。
% 電極に対しての重み付け係数(空間フィルタ)$w \in \mathbb{R}^M$を何らかの方法で決定することで、
% \begin{equation*}
%     z = w^T X \in \mathbb{R}^N    
% \end{equation*}
% と変換する。次に$z$に対して短時間フーリエ変換${\cal F}(\cdot)$を行い、
% \begin{equation*}
%     A = {\cal F}(z) \in \mathbb{R}^{F\times N}
% \end{equation*}
% を獲得する。短時間フーリエ変換の代わりにウェーブレット変換などの時間周波数解析を用いる場合もある。
% スペクトログラム$A$から事象関連脱同期を見出すことができれば、ある特定の周波数帯域でのパワーを見張ることで運動の意図推定が可能となる。
% 実際、特定の周波数帯域のパワーに対して閾値を設けることで、運動意図推定を行う手法も提案されている。
% ただし、事象関連脱同期が生じる周波数帯域には個人差があることが想定されるため、スペクトログラム$A$に対して何らかの行列分解を用いて特徴量を取り出すことも提案されている。



% \subsection{Convolutional Neural Networks(ConvNets)}
% この手法に関しては\cite{deepconv}を参照されたい。