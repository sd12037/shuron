
\begin{center}
    \huge{数学的表記について}
\end{center}

本文中で用いられる数学的な表記の中で、
慣習的に用いているものを記載する。
ここに記載した数式は再度本文中で明示する場合もあるが、
特に誤解を招かない場合には予告なしに用いる。

\begin{align}
    & x&& ベクトル(スカラーを表す場合には本文中で明記して用いる)\\
    & x^T && 転置記号\\
    & |x| && ユークリッドノルム\\
    & |x|_p && l_pノルム\\
    & X&& 行列(他の用途の場合には明記する)\\
    & X^{-1} && 逆行列記号 \\
    & \mathbb R && 実数集合\\
    & \mathbb R^m && m次元実数集合\\
    & \mathbb R^{m\times n} && m次元実数集合とn次元実数集合の直積集合\\
    & \mathbb I(\cdot) && 指示関数 \\
    & a\in A && aは集合Aの元\\
    & B\subset A && Bは集合Aの真部分集合\\
    & B\subseteq A && Bは集合Aの部分集合\\
    & \pi && 円周率\\
    & e && ネイピア数\\    
    & \log(\cdot) && ネイピア数を底とした対数関数\\
    & p(\cdot) && 確率密度関数あるいは確率質量関数\\
    & q(\cdot) && 確率密度関数あるいは確率質量関数\\
    & p(x_1,x_2) && x_1,x_2の同時確率\\
    & p(x_1\mid x_2) && x_2に条件付けられたx_1の条件付き確率\\
    & {\cal N}(\mu,\sigma^2) && スカラーの平均\mu、分散\sigma^2の1次元ガウス分布\\
    & {\cal N}(\mu,\Sigma) && 平均ベクトル\mu 、分散共分散行列 \Sigma の多次元ガウス分布\\
    & \mathbb E[\cdot] && 期待値演算\\
    & \mathbb E[x_1 \mid x_2] && x_2に条件付けられたx_1の条件付き期待値\\
    & {\cal KL}(p\mid q) && pからqへのカルバック・ライブラーダイバージェンス\\
    & \cal D && データ集合
\end{align}
