
\begin{center}
    \huge{数学的表記について}
\end{center}

% 本文中で用いられる数学的な表記の中で、
% 慣習的に用いているものを記載する。
% ここに記載した数式は再度本文中で明示する場合もあるが、
% 特に誤解を招かない場合には予告なしに用いる。
\begin{align}
    & x && ベクトル(スカラーを表す場合には本文中で明記して用いる)\\
    & |x| && ユークリッドノルム\\
    & |x|_p && l_pノルム\\
    & X&& 行列(他の用途の場合には明記する)\\
    & {\cdot}^T && 転置記号\\
    & {\cdot}^{-1} && 逆行列記号 \\
    & I && 単位行列 \\
    & {\rm diag}(x) && ベクトルxの要素を対角成分に持つ対角行列 \\
    % & {\rm diag}(\ram_1, \ram2, \cdots) & スカラー\ram_1, \lam_2, \cdotsを対角成分に持つ対角行列 \\
    %集合
    & \mathbb R && 実数集合\\
    & \mathbb R^m && m次元実数集合\\
    & \mathbb R^{m\times n} && m次元実数集合とn次元実数集合の直積集合\\
    & \mathbb I(S) && 指示関数(Sが真の時に1を返し、その他の場合に0を返す) \\
    & \{a,b\} && aとbを元に持つ集合\\
    & (a,b) && a<bでaとbを端点とする閉区間\\
    & [a,b] && a<bでaとbを端点とする開区間\\
    & a\in A && aは集合Aの元\\
    & B\subset A && Bは集合Aの真部分集合\\
    & B\subseteq A && Bは集合Aの部分集合\\
    %初等関数
    & \pi && 円周率\\
    & e && ネイピア数\\    
    & \log(\cdot) && ネイピア数を底とした対数関数\\
    & \max_xf(w,x) && xに関するfの最大値\\
    & \min_xf(w,x) && xに関するfの最小値\\
    & \argmax_xf(w,x) && xに関してfが最大となるときのx\\
    & \argmin_xf(w,x) && xに関してfが最小となるときのx\\
    & (f\circ g)(\cdot) && f(\cdot)とg(\cdot)の合成関数f(g(\cdot)) \\
    & {\nabla} && ナブラ記号 \\
    & {\nabla}_w L(w,x) && L(w,x)のwに関する勾配\\
    %確率統計
    & p(\cdot) && 確率密度関数あるいは確率質量関数(慣習的に確率分布と表記する)\\
    & q(\cdot) && 確率密度関数あるいは確率質量関数(慣習的に確率分布と表記する)\\
    & p(x_1,x_2) && x_1,x_2の同時確率\\
    & p(x_1\mid x_2) && x_2に条件付けられたx_1の条件付き確率\\
    & {\cal N}(\mu,\sigma^2) && 平均\mu、分散\sigma^2の1次元ガウス分布\\
    & {\cal N}(\mu,\Sigma) && 平均ベクトル\mu 、分散共分散行列 \Sigma の多次元ガウス分布\\
    & {\rm Bern}(p) && pの確率で1を、(1-p)の確率で0を生成するベルヌーイ分布\\
    & \mathbb E[\cdot] && 期待値演算\\
    & \mathbb E[x_1 \mid x_2] && x_2に条件付けられたx_1の条件付き期待値\\
    & \mathbb E_{X \in C}[f(X)] && 集合Cに属するXによるf(X)の期待値\\
    & \mathbb E_{p(x)}[f(x)] && 確率密度関数p(x)によるf(x)の期待値\\
    & {\cal KL}(p\mid q) && pからqへのカルバック・ライブラーダイバージェンス\\
    & \rm i.i.d. && 独立同分布から生起することを表す\\
    %機械学習
    & \cal D && データ集合 \\
    & x_{a:b} && x_a,x_{a+1},\cdots,x_b
\end{align}
